\documentclass{article}

\usepackage{fancyhdr}
\usepackage{extramarks}
\usepackage{amsmath}
\usepackage{amsthm}
\usepackage{amsfonts}
\usepackage{tikz}
\usepackage[plain]{algorithm}
\usepackage{algpseudocode}
\usepackage{qcircuit}

\usetikzlibrary{automata,positioning}

%
% Basic Document Settings
%

\topmargin=-0.45in
\evensidemargin=0in
\oddsidemargin=0in
\textwidth=6.5in
\textheight=9.0in
\headsep=0.25in

\linespread{1.1}

\pagestyle{fancy}
\lhead{\hmwkAuthorName}
\chead{}
\rhead{\hmwkClass}
\lfoot{\lastxmark}
\cfoot{\thepage}

\renewcommand\headrulewidth{0.4pt}
\renewcommand\footrulewidth{0.4pt}

\setlength\parindent{0pt}

%
% Create Problem Sections
%

\newcommand{\enterProblemHeader}[1]{
    \nobreak\extramarks{}{Problem \arabic{#1} continued on next page\ldots}\nobreak{}
    \nobreak\extramarks{Problem \arabic{#1} (continued)}{Problem \arabic{#1} continued on next page\ldots}\nobreak{}
}

\newcommand{\exitProblemHeader}[1]{
    \nobreak\extramarks{Problem \arabic{#1} (continued)}{Problem \arabic{#1} continued on next page\ldots}\nobreak{}
    \stepcounter{#1}
    \nobreak\extramarks{Problem \arabic{#1}}{}\nobreak{}
}

\setcounter{secnumdepth}{0}
\newcounter{partCounter}
\newcounter{homeworkProblemCounter}
\setcounter{homeworkProblemCounter}{1}
\nobreak\extramarks{Problem \arabic{homeworkProblemCounter}}{}\nobreak{}

%
% Homework Problem Environment
%
% This environment takes an optional argument. When given, it will adjust the
% problem counter. This is useful for when the problems given for your
% assignment aren't sequential. See the last 3 problems of this template for an
% example.
%
\newenvironment{homeworkProblem}[1][-1]{
    \ifnum#1>0
        \setcounter{homeworkProblemCounter}{#1}
    \fi
    \section{Problem \arabic{homeworkProblemCounter}}
    \setcounter{partCounter}{1}
    \enterProblemHeader{homeworkProblemCounter}
}{
    \exitProblemHeader{homeworkProblemCounter}
}

%
% Homework Details
%   - Title
%   - Due date
%   - Class
%   - Section/Time
%   - Instructor
%   - Author
%

\newcommand{\hmwkTitle}{Homework\ \#1}
\newcommand{\hmwkDueDate}{April 10, 2020}
\newcommand{\hmwkClass}{Intorduction to Quantum Computing}
\newcommand{\hmwkAuthorName}{Jakub Filipek}

%
% Title Page
%

\title{
    % \vspace{in}
    \textmd{\textbf{\hmwkClass:\ \hmwkTitle}}\\
    \normalsize\vspace{0.1in}\small{Due\ on\ \hmwkDueDate\ at 3:10pm}\\
}

\author{\hmwkAuthorName}
\date{}

\renewcommand{\part}[1]{\textbf{\large Part \Alph{partCounter}}\stepcounter{partCounter}\\}

%
% Various Helper Commands
%

% Useful for algorithms
\newcommand{\alg}[1]{\textsc{\bfseries \footnotesize #1}}

% For derivatives
\newcommand{\deriv}[1]{\frac{\mathrm{d}}{\mathrm{d}x} (#1)}

% For partial derivatives
\newcommand{\pderiv}[2]{\frac{\partial}{\partial #1} (#2)}

% Integral dx
\newcommand{\dx}{\mathrm{d}x}

% Alias for the Solution section header
\newcommand{\solution}{\textbf{\large Solution}}

% Probability commands: Expectation, Variance, Covariance, Bias
\newcommand{\E}{\mathrm{E}}
\newcommand{\Var}{\mathrm{Var}}
\newcommand{\Cov}{\mathrm{Cov}}
\newcommand{\Bias}{\mathrm{Bias}}

\newcommand{\norm}[1]{\left\lVert#1\right\rVert}

\newcommand{\bra}[1]{\lstick#1|}
\newcommand{\ket}[1]{|#1\rangle}
\newcommand{\qbra}{\bra{\psi}}
\newcommand{\qket}{\ket{\psi}}


\newcommand{\qwxo}[2][-1]{\ar @{-} [#1,0]|*+<2pt,4pt>[Fo]{#2}}

\begin{document}

\maketitle

% \pagebreak

\begin{homeworkProblem}

    \begin{figure}[h]
        \[\Qcircuit @C=1em @R=1em {
                & \qswap      & \gate{H} & \gate{\sqrt{X}} & \qw \\
                & \qswap \qwx & \gate{H} & \ctrl{-1}       & \qw
            }
        \]
        \caption{Simplified Circuit for Problem 1}
        \label{circ:p1}
    \end{figure}

    \subsection*{Part (a)}

    Let us look at Fig.~\ref{circ:p1}. The matrices for given operations are as follow:
    \begin{align*}
        \text{SWAP} &= \begin{pmatrix}
            1 & 0 & 0 & 0 \\
            0 & 0 & 1 & 0 \\
            0 & 1 & 0 & 0 \\
            0 & 0 & 0 & 1 \\
        \end{pmatrix} \\
        H \otimes I &= \frac{1}{\sqrt{2}}\begin{pmatrix}
            1 & 0 & 1  & 0  \\
            0 & 1 & 0  & 1  \\
            1 & 0 & -1 & 0  \\
            0 & 1 & 0  & -1 \\
        \end{pmatrix} \\
        I \otimes H &= \frac{1}{\sqrt{2}}\begin{pmatrix}
            1 & 1  & 0 & 0  \\
            1 & -1 & 0 & 0  \\
            0 & 0  & 1 & 1  \\
            0 & 0  & 1 & -1 \\
        \end{pmatrix} \\
        \text{C} \sqrt{X} &= \begin{pmatrix}
            1 & 0                & 0 & 0               \\
            0 & \frac{1 + i}{2}  & 0 & \frac{1 - i}{2} \\
            0 & 0                & 1 & 0               \\
            0 & \frac{1 - i}{2}  & 0 & \frac{1 + i}{2} \\
        \end{pmatrix} \\
    \end{align*}

    We can see that SWAP and C$\sqrt{X}$ are already two-level unitaries.

    For both Hadamard gates:
    \begin{align*}
        U_1 &= \begin{pmatrix}
            \frac{1}{\sqrt{2}} & 0 & \frac{1}{\sqrt{2}} & 0 \\
            0 & 1 & 0 & 0 \\
            \frac{1}{\sqrt{2}} & 0 & -\frac{1}{\sqrt{2}} & 0 \\
            0 & 0 & 0 & 1 \\
        \end{pmatrix} \\
        U_2 = U_1(H \otimes I) &= \frac{1}{\sqrt{2}}
        \begin{pmatrix}
            \frac{1}{\sqrt{2}} & 0 & \frac{1}{\sqrt{2}} & 0 \\
            0 & 1 & 0 & 0 \\
            \frac{1}{\sqrt{2}} & 0 & -\frac{1}{\sqrt{2}} & 0 \\
            0 & 0 & 0 & 1 \\
        \end{pmatrix}\begin{pmatrix}
            1 & 0 & 1  & 0  \\
            0 & 1 & 0  & 1  \\
            1 & 0 & -1 & 0  \\
            0 & 1 & 0  & -1 \\
        \end{pmatrix} \\
        &= \begin{pmatrix}
            1 & 0 & 0 & 0  \\
            0 & \frac{1}{\sqrt{2}} & 0 & \frac{1}{\sqrt{2}}  \\
            0 & 0 & 1 & 0  \\
            0 & \frac{1}{\sqrt{2}} & 0 & -\frac{1}{\sqrt{2}} \\
        \end{pmatrix} \\
        U_3 &= \begin{pmatrix}
            \frac{1}{\sqrt{2}} & \frac{1}{\sqrt{2}} & 0 & 0  \\
            \frac{1}{\sqrt{2}} & -\frac{1}{\sqrt{2}} & 0 & 0  \\
            0 & 0  & 1 & 0  \\
            0 & 0  & 0 & 1 \\
        \end{pmatrix} \\
        U_4 = U_3(I \otimes H) &= \frac{1}{\sqrt{2}}
        \begin{pmatrix}
            \frac{1}{\sqrt{2}} & \frac{1}{\sqrt{2}} & 0 & 0  \\
            \frac{1}{\sqrt{2}} & -\frac{1}{\sqrt{2}} & 0 & 0  \\
            0 & 0  & 1 & 0  \\
            0 & 0  & 0 & 1 \\
        \end{pmatrix}\begin{pmatrix}
            1 & 1  & 0 & 0  \\
            1 & -1 & 0 & 0  \\
            0 & 0  & 1 & 1  \\
            0 & 0  & 1 & -1 \\
        \end{pmatrix} \\
        &= \begin{pmatrix}
            1 & 0 & 0 & 0  \\
            0 & 1 & 0 & 0  \\
            0 & 0 & 1 & 0  \\
            0 & 0 & 0 & -1 \\
        \end{pmatrix}
    \end{align*}

    Hence the original matrix can be decomposed as:
    \begin{equation*}
        \text{C}(\sqrt{X})(U_4^\dagger U_3^\dagger U_2^\dagger U_1^\dagger)\text{SWAP}
    \end{equation*}

    \subsection*{Part (b)}

    Let us look again at a circuit on Fig.~\ref{circ:p1}.
    Swap can be expressed as three CNOT (and hence Toffoli) gates.

    \paragraph*{}

    Then the controlled \textit{square-root-of-not} can be expressed as:
    \[\Qcircuit @C=1em @R=2em {
            & \ctrl{1} & \qswap                                                        & \ctrl{1} & \qw \\
            & \targ    & \qswap \qwxo{\scalebox{0.5}{$1\hspace{-1pt}/\hspace{-1pt}2$}} & \targ    & \qw
        }
    \]

    Where $\sqrt{SWAP}$ gate can be expressed as series of CNOTs, $R_Z$ and $R_Y$ rotations

\end{homeworkProblem}

\newpage
\begin{homeworkProblem}

\end{homeworkProblem}

\newpage
\begin{homeworkProblem}
    \subsection*{Part (a)}

    \begin{align*}
        \qket &= \sqrt{\frac{e-1}{e}} \sum\limits_{x=0}^\infty e^{-x/2}\ket{x} \\
            &= \sqrt{\frac{e-1}{e}} (\sum\limits_{x=0}^\infty e^{-x}\ket{2x} + \sum\limits_{x=0}^\infty e^{-x - 1/2}\ket{2x+1}) \\
            &= \sqrt{\frac{e-1}{e}} (\sum\limits_{x=0}^\infty e^{-x}\ket{2x} + e^{-1/2}\sum\limits_{x=0}^\infty e^{-x}\ket{2x+1}) \\
    \end{align*}

    Then we see that the ratio of probabilities of even and odd numbers is:
    \begin{equation*}
        \frac{P(\text{even})}{P(\text{odd})} = \frac{1^2}{(e^{-1/2})^2} = e
    \end{equation*}

    Then from probability:
    \begin{align*}
        1 &= P(\text{all}) = P(\text{odd}) + P(\text{even}) \\
          &= P(\text{even})\frac{P(\text{odd})}{P(\text{even})} + P(\text{even}) \\
          &= P(\text{even})e^{-1} + P(\text{even}) \\
          &= P(\text{even})(1 + \frac{1}{e}) = 1
    \end{align*}

    Hence:
    \begin{align*}
        P(\text{even})(1 + \frac{1}{e}) = 1 \\
        P(\text{even}) = \frac{1}{(1 + \frac{1}{e})} \\
        P(\text{even}) = \frac{e}{(e + 1)}
    \end{align*}

    \subsection*{Part (b)}

\end{homeworkProblem}
\end{document}