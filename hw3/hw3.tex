\documentclass{article}

\usepackage{fancyhdr}
\usepackage{extramarks}
\usepackage{amsmath}
\usepackage{amsthm}
\usepackage{amsfonts}
\usepackage{tikz}
\usepackage[plain]{algorithm}
\usepackage{algpseudocode}
\usepackage{qcircuit}

\usetikzlibrary{automata,positioning}

%
% Basic Document Settings
%

\topmargin=-0.45in
\evensidemargin=0in
\oddsidemargin=0in
\textwidth=6.5in
\textheight=9.0in
\headsep=0.25in

\linespread{1.1}

\pagestyle{fancy}
\lhead{\hmwkAuthorName}
\chead{}
\rhead{\hmwkClass}
\lfoot{\lastxmark}
\cfoot{\thepage}

\renewcommand\headrulewidth{0.4pt}
\renewcommand\footrulewidth{0.4pt}

\setlength\parindent{0pt}

%
% Create Problem Sections
%

\newcommand{\enterProblemHeader}[1]{
    \nobreak\extramarks{}{Problem \arabic{#1} continued on next page\ldots}\nobreak{}
    \nobreak\extramarks{Problem \arabic{#1} (continued)}{Problem \arabic{#1} continued on next page\ldots}\nobreak{}
}

\newcommand{\exitProblemHeader}[1]{
    \nobreak\extramarks{Problem \arabic{#1} (continued)}{Problem \arabic{#1} continued on next page\ldots}\nobreak{}
    \stepcounter{#1}
    \nobreak\extramarks{Problem \arabic{#1}}{}\nobreak{}
}

\setcounter{secnumdepth}{0}
\newcounter{partCounter}
\newcounter{homeworkProblemCounter}
\setcounter{homeworkProblemCounter}{1}
\nobreak\extramarks{Problem \arabic{homeworkProblemCounter}}{}\nobreak{}

%
% Homework Problem Environment
%
% This environment takes an optional argument. When given, it will adjust the
% problem counter. This is useful for when the problems given for your
% assignment aren't sequential. See the last 3 problems of this template for an
% example.
%
\newenvironment{homeworkProblem}[1][-1]{
    \ifnum#1>0
        \setcounter{homeworkProblemCounter}{#1}
    \fi
    \section{Problem \arabic{homeworkProblemCounter}}
    \setcounter{partCounter}{1}
    \enterProblemHeader{homeworkProblemCounter}
}{
    \exitProblemHeader{homeworkProblemCounter}
}

%
% Homework Details
%   - Title
%   - Due date
%   - Class
%   - Section/Time
%   - Instructor
%   - Author
%

\newcommand{\hmwkTitle}{Homework\ \#1}
\newcommand{\hmwkDueDate}{April 28, 2020}
\newcommand{\hmwkClass}{Intorduction to Quantum Computing}
\newcommand{\hmwkAuthorName}{Jakub Filipek}

%
% Title Page
%

\title{
    % \vspace{in}
    \textmd{\textbf{\hmwkClass:\ \hmwkTitle}}\\
    \normalsize\vspace{0.1in}\small{Due\ on\ \hmwkDueDate}\\
}

\author{\hmwkAuthorName}
\date{}

\renewcommand{\part}[1]{\textbf{\large Part \Alph{partCounter}}\stepcounter{partCounter}\\}

%
% Various Helper Commands
%

% Useful for algorithms
\newcommand{\alg}[1]{\textsc{\bfseries \footnotesize #1}}

% For derivatives
\newcommand{\deriv}[1]{\frac{\mathrm{d}}{\mathrm{d}x} (#1)}

% For partial derivatives
\newcommand{\pderiv}[2]{\frac{\partial}{\partial #1} (#2)}

% Integral dx
\newcommand{\dx}{\mathrm{d}x}

% Alias for the Solution section header
\newcommand{\solution}{\textbf{\large Solution}}

% Probability commands: Expectation, Variance, Covariance, Bias
\newcommand{\E}{\mathrm{E}}
\newcommand{\Var}{\mathrm{Var}}
\newcommand{\Cov}{\mathrm{Cov}}
\newcommand{\Bias}{\mathrm{Bias}}

\newcommand{\norm}[1]{\left\lVert#1\right\rVert}

\newcommand{\bra}[1]{\lstick#1|}
\newcommand{\ket}[1]{|#1\rangle}
\newcommand{\qbra}{\bra{\psi}}
\newcommand{\qket}{\ket{\psi}}


\newcommand{\qwxo}[2][-1]{\ar @{-} [#1,0]|*+<2pt,4pt>[Fo]{#2}}

\begin{document}

\maketitle

% \pagebreak

\begin{homeworkProblem}

    \subsection*{Part (a)}

    \begin{figure}[h]
        \[\Qcircuit @C=1em @R=1em {
                & \qswap      & \gate{H} & \gate{\sqrt{X}} & \qw \\
                & \qswap \qwx & \gate{H} & \ctrl{-1}       & \qw
            }
        \]
        \caption{Simplified Circuit for Problem 1a}
        \label{circ:p1}
    \end{figure}

    Let us look at Fig.~\ref{circ:p1}. The matrices for given operations are as follow:
    \begin{align*}
        \text{SWAP} &= \begin{pmatrix}
            1 & 0 & 0 & 0 \\
            0 & 0 & 1 & 0 \\
            0 & 1 & 0 & 0 \\
            0 & 0 & 0 & 1 \\
        \end{pmatrix} \\
        H \otimes I &= \frac{1}{\sqrt{2}}\begin{pmatrix}
            1 & 0 & 1  & 0  \\
            0 & 1 & 0  & 1  \\
            1 & 0 & -1 & 0  \\
            0 & 1 & 0  & -1 \\
        \end{pmatrix} \\
        I \otimes H &= \frac{1}{\sqrt{2}}\begin{pmatrix}
            1 & 1  & 0 & 0  \\
            1 & -1 & 0 & 0  \\
            0 & 0  & 1 & 1  \\
            0 & 0  & 1 & -1 \\
        \end{pmatrix} \\
        \text{C} \sqrt{X} &= \begin{pmatrix}
            1 & 0                & 0 & 0               \\
            0 & \frac{1 + i}{2}  & 0 & \frac{1 - i}{2} \\
            0 & 0                & 1 & 0               \\
            0 & \frac{1 - i}{2}  & 0 & \frac{1 + i}{2} \\
        \end{pmatrix} \\
    \end{align*}

    We can see that SWAP and C$\sqrt{X}$ are already two-level unitaries.

    For both Hadamard gates:
    \begin{align*}
        U_1 &= \begin{pmatrix}
            \frac{1}{\sqrt{2}} & 0 & \frac{1}{\sqrt{2}} & 0 \\
            0 & 1 & 0 & 0 \\
            \frac{1}{\sqrt{2}} & 0 & -\frac{1}{\sqrt{2}} & 0 \\
            0 & 0 & 0 & 1 \\
        \end{pmatrix} \\
        U_2 = U_1(H \otimes I) &= \frac{1}{\sqrt{2}}
        \begin{pmatrix}
            \frac{1}{\sqrt{2}} & 0 & \frac{1}{\sqrt{2}} & 0 \\
            0 & 1 & 0 & 0 \\
            \frac{1}{\sqrt{2}} & 0 & -\frac{1}{\sqrt{2}} & 0 \\
            0 & 0 & 0 & 1 \\
        \end{pmatrix}\begin{pmatrix}
            1 & 0 & 1  & 0  \\
            0 & 1 & 0  & 1  \\
            1 & 0 & -1 & 0  \\
            0 & 1 & 0  & -1 \\
        \end{pmatrix} \\
        &= \begin{pmatrix}
            1 & 0 & 0 & 0  \\
            0 & \frac{1}{\sqrt{2}} & 0 & \frac{1}{\sqrt{2}}  \\
            0 & 0 & 1 & 0  \\
            0 & \frac{1}{\sqrt{2}} & 0 & -\frac{1}{\sqrt{2}} \\
        \end{pmatrix} \\
        U_3 &= \begin{pmatrix}
            \frac{1}{\sqrt{2}} & \frac{1}{\sqrt{2}} & 0 & 0  \\
            \frac{1}{\sqrt{2}} & -\frac{1}{\sqrt{2}} & 0 & 0  \\
            0 & 0  & 1 & 0  \\
            0 & 0  & 0 & 1 \\
        \end{pmatrix} \\
        U_4 = U_3(I \otimes H) &= \frac{1}{\sqrt{2}}
        \begin{pmatrix}
            \frac{1}{\sqrt{2}} & \frac{1}{\sqrt{2}} & 0 & 0  \\
            \frac{1}{\sqrt{2}} & -\frac{1}{\sqrt{2}} & 0 & 0  \\
            0 & 0  & 1 & 0  \\
            0 & 0  & 0 & 1 \\
        \end{pmatrix}\begin{pmatrix}
            1 & 1  & 0 & 0  \\
            1 & -1 & 0 & 0  \\
            0 & 0  & 1 & 1  \\
            0 & 0  & 1 & -1 \\
        \end{pmatrix} \\
        &= \begin{pmatrix}
            1 & 0 & 0 & 0  \\
            0 & 1 & 0 & 0  \\
            0 & 0 & 1 & 0  \\
            0 & 0 & 0 & -1 \\
        \end{pmatrix}
    \end{align*}

    Hence the original matrix can be decomposed as:
    \begin{equation*}
        \text{C}(\sqrt{X})(U_4^\dagger U_3^\dagger U_2^\dagger U_1^\dagger)\text{SWAP}
    \end{equation*}

    \subsection*{Part (b)}

    \begin{figure}[h]
        \[\Qcircuit @C=1em @R=1em {
                & \qswap      & \qw      & \ctrl{1} & \gate{H} & \qw \\
                & \qswap \qwx & \gate{H} & \gate{S} & \qw      & \qw
            }
        \]
        \caption{Simplified Circuit for Problem 1b}
    \end{figure}

    The above circuit also evaluates to the original matrix.

    Firstly, let's note that SWAP gate can be decomposed into 3 Toffoli's. Then, the Controlled S gate can be written as:
    \[\Qcircuit @C=1em @R=1em {
        \lstick{\ket{0}} & \targ     & \gate{R_Z(\frac{\pi}{2})}  & \targ     & \qw & \rstick{\ket{0}} \\
                         & \ctrl{-1} & \qw                        & \ctrl{-1} & \qw &                  \\
                         & \ctrl{-2} & \qw                        & \ctrl{-2} & \qw &
      }
    \]

    Where $HR_Z(\pi)H = HZH = X$, and $R_Z(\frac{\pi}{2}) = S$.

    Overall then a whole circuit can be written as:

    \begin{figure}[h]
        \[\Qcircuit @C=1em @R=1em {
            \lstick{\ket{0}} & \qw      & \qw             & \qw      & \qw       & \qw      & \qw       & \targ     & \gate{R_Z(\frac{\pi}{2})}  & \targ     & \qw      & \qw             & \qw      & \qw & \rstick{\ket{0}} \\
            \lstick{\ket{0}} & \gate{H} & \gate{R_Z(\pi)} & \gate{H} & \ctrl{1}  & \ctrl{2} & \ctrl{1}  & \qw       & \qw                        & \qw       & \gate{H} & \gate{R_Z(\pi)} & \gate{H} & \qw & \rstick{\ket{0}} \\
                             & \qw      & \qw             & \qw      & \targ     & \ctrl{1} & \targ     & \ctrl{-2} & \qw                        & \ctrl{-2} & \gate{H} & \qw             & \qw      & \qw & \\
                             & \qw      & \qw             & \gate{H} & \ctrl{-1} & \targ    & \ctrl{-1} & \ctrl{-3} & \qw                        & \ctrl{-3} & \qw      & \qw             & \qw      & \qw &
            }
        \]
        \caption{Full Circuit for Problem 1b}
    \end{figure}

    \paragraph*{}

    This circuit contains:
    \begin{itemize}
        \item 6 Hadamard gates
        \item 3 $R_Z$ gates
        \item 5 Toffoli gates
    \end{itemize}

    While it is not the most optimal solution (i.e. there exists a solution with only one ancilla qubit), it works.
\end{homeworkProblem}

\vspace{2cm}
\begin{homeworkProblem}

    Let us create an $R_Z(\theta)$ gate.

    \begin{figure}[h]
        \[\Qcircuit @C=1em @R=1em {
            & \ctrl{1}            & \gate{X} & \ctrl{1}           & \gate{X} & \qw \\
            & \gate{R_X(-\theta)} & \qw      & \gate{R_X(\theta)} & \qw      & \qw
        }\]
    \end{figure}

    Let us analyze the above circuit:
    \begin{align*}
        \begin{pmatrix}
            0 & 0 & 1 & 0 \\
            0 & 0 & 0 & 1 \\
            1 & 0 & 0 & 0 \\
            0 & 1 & 0 & 0 \\
        \end{pmatrix}
        \begin{pmatrix}
            1 & 0 & 0 & 0 \\
            0 & 1 & 0 & 0 \\
            0 & 0 & \cos(\frac{\theta}{2}) & -i\sin(\frac{\theta}{2}) \\
            0 & 0 & -i\sin(\frac{\theta}{2}) & \cos(\frac{\theta}{2}) \\
        \end{pmatrix}
        \begin{pmatrix}
            0 & 0 & 1 & 0 \\
            0 & 0 & 0 & 1 \\
            1 & 0 & 0 & 0 \\
            0 & 1 & 0 & 0 \\
        \end{pmatrix}
        \begin{pmatrix}
            1 & 0 & 0 & 0 \\
            0 & 1 & 0 & 0 \\
            0 & 0 & \cos(\frac{-\theta}{2}) & -i\sin(\frac{-\theta}{2}) \\
            0 & 0 & -i\sin(\frac{-\theta}{2}) & \cos(\frac{-\theta}{2}) \\
        \end{pmatrix} = \\
        \begin{pmatrix}
            0 & 0 & \cos(\frac{\theta}{2}) & -i\sin(\frac{\theta}{2}) \\
            0 & 0 & -i\sin(\frac{\theta}{2}) & \cos(\frac{\theta}{2}) \\
            1 & 0 & 0 & 0 \\
            0 & 1 & 0 & 0 \\
        \end{pmatrix}
        \begin{pmatrix}
            0 & 0 & \cos(\frac{-\theta}{2}) & -i\sin(\frac{-\theta}{2}) \\
            0 & 0 & -i\sin(\frac{-\theta}{2}) & \cos(\frac{-\theta}{2}) \\
            1 & 0 & 0 & 0 \\
            0 & 1 & 0 & 0 \\
        \end{pmatrix} = \\
        \begin{pmatrix}
            \cos(\frac{\theta}{2}) & -i\sin(\frac{\theta}{2}) & 0 & 0 \\
            -i\sin(\frac{\theta}{2}) & \cos(\frac{\theta}{2}) & 0 & 0 \\
            0 & 0 & \cos(\frac{-\theta}{2}) & -i\sin(\frac{-\theta}{2}) \\
            0 & 0 & -i\sin(\frac{-\theta}{2}) & \cos(\frac{-\theta}{2}) \\
        \end{pmatrix}
    \end{align*}


    Let us consider what happens when a $\ket{0+}$ state is applied:
    \begin{align*}
        \ket{0+} = \frac{1}{\sqrt{2}} (\ket{00} + \ket{01}) &\rightarrow \\
        \frac{1}{\sqrt{2}} ((\cos(\frac{\theta}{2})\ket{00} - i\sin(\frac{\theta}{2}) \ket{01}) + (- i\sin(\frac{\theta}{2})\ket{00} + \cos(\frac{\theta}{2})\ket{01})) &= \\
        \frac{1}{\sqrt{2}} ((\cos(\frac{\theta}{2})- i\sin(\frac{\theta}{2})) \ket{00} + (\cos(\frac{\theta}{2}) - i\sin(\frac{\theta}{2})) \ket{01}) &= \\
        \frac{1}{\sqrt{2}} (e^{-i\frac{\theta}{2}} \ket{00} + e^{-i\frac{\theta}{2}} \ket{01}) &= \\
        \frac{e^{-i\frac{\theta}{2}}}{\sqrt{2}} (\ket{00} + \ket{01}) &= \\
        \frac{e^{-i\frac{\theta}{2}}}{\sqrt{2}} \ket{0+}
    \end{align*}

    On the other hand if $\ket{1+}$ is applied then:
    \begin{align*}
        \ket{1+} = \frac{1}{\sqrt{2}} (\ket{10} + \ket{11}) &\rightarrow \\
        \frac{1}{\sqrt{2}} ((\cos(\frac{-\theta}{2})\ket{10} - i\sin(\frac{-\theta}{2}) \ket{11}) + (- i\sin(\frac{-\theta}{2})\ket{10} + \cos(\frac{-\theta}{2})\ket{11})) &= \\
        \frac{1}{\sqrt{2}} ((\cos(\frac{-\theta}{2})- i\sin(\frac{-\theta}{2})) \ket{10} + (\cos(\frac{-\theta}{2}) - i\sin(\frac{-\theta}{2})) \ket{11}) &= \\
        \frac{1}{\sqrt{2}} (e^{i\frac{\theta}{2}} \ket{10} + e^{i\frac{\theta}{2}} \ket{11}) &= \\
        \frac{e^{i\frac{\theta}{2}}}{\sqrt{2}} (\ket{10} + \ket{11}) &= \\
        \frac{e^{i\frac{\theta}{2}}}{\sqrt{2}} \ket{1+}
    \end{align*}

    Hence we can see that for a first qubit this is equivalent to applying a $R_Z(\theta)$ gate.

    Now we have to dicuss how to create a $\ket{+}$ state. First let's not that all $R_X(\theta)$, CNOT, Toffoli, and X gates
    can be created from this controlled $R_X(\theta)$ gate (by setting controls to $\ket{1}$ and/or by specific arguments $\theta$).

    The following circuit:
    \[\Qcircuit @C=1em @R=1em {
        \lstick{\ket{1}} & \gate{R_X(\frac{\pi}{2})} & \ctrl{1} & \gate{R_X(-\frac{\pi}{2})} & \qw \\
        \lstick{\ket{0}} & \qw                       & \targ    & \qw                        & \qw
    }\]

    This corresponds to following matrix multiplication:
    \begin{align*}
        \frac{1}{\sqrt{2}}
        \begin{pmatrix}
            1 & 0 & i & 0 \\
            0 & 1 & 0 & i \\
            i & 0 & 1 & 0 \\
            0 & i & 0 & 1 \\
        \end{pmatrix}
        \begin{pmatrix}
            1 & 0 & 0 & 0 \\
            0 & 1 & 0 & 0 \\
            0 & 0 & 0 & 1 \\
            0 & 0 & 1 & 0 \\
        \end{pmatrix}
        \frac{1}{\sqrt{2}}
        \begin{pmatrix}
            1  & 0  & -i & 0  \\
            0  & 1  & 0  & -i \\
            -i & 0  & 1  & 0  \\
            0  & -i & 0  & 1  \\
        \end{pmatrix}
        \begin{pmatrix}
            0 \\ 0 \\ 1 \\ 0
        \end{pmatrix} = \\
        \frac{1}{2}
        \begin{pmatrix}
            1 & 0 & i & 0 \\
            0 & 1 & 0 & i \\
            i & 0 & 1 & 0 \\
            0 & i & 0 & 1 \\
        \end{pmatrix}
        \begin{pmatrix}
            1 & 0 & 0 & 0 \\
            0 & 1 & 0 & 0 \\
            0 & 0 & 0 & 1 \\
            0 & 0 & 1 & 0 \\
        \end{pmatrix}
        \begin{pmatrix}
            -i \\ 0 \\ 1 \\ 0
        \end{pmatrix} = \\
        \frac{1}{2}
        \begin{pmatrix}
            1 & 0 & i & 0 \\
            0 & 1 & 0 & i \\
            i & 0 & 1 & 0 \\
            0 & i & 0 & 1 \\
        \end{pmatrix}
        \begin{pmatrix}
            -i \\ 0 \\ 0 \\ 1
        \end{pmatrix} = \\
        \frac{1}{2}
        \begin{pmatrix}
            -i \\ i \\ 1 \\ 1
        \end{pmatrix}
    \end{align*}

    Now, we can measure first qubit until we get $\ket{1}$ state. In such case, the second will be in $\ket{+}$ which is required for the implementation of the $R_Z(\theta)$ gate.

    \paragraph{}
    Since, as shown in a class a sequence of $R_Z$ and $R_X$, can implement any 1-qubit unitary gate they are sufficient to be universal for single qubit operations.
    Additionally, if we include a Toffoli, we can then implement any Controlled Unitary operation.
    Hence $R_Z, R_X$ and $CNOT$ is a universal set. Hence the doubly controlled $R_X$ gate is also universal.

\end{homeworkProblem}

\vspace{2cm}
\begin{homeworkProblem}
    \subsection*{Part (a)}

    \begin{align*}
        \qket &= \sqrt{\frac{e-1}{e}} \sum\limits_{x=0}^\infty e^{-x/2}\ket{x} \\
            &= \sqrt{\frac{e-1}{e}} (\sum\limits_{x=0}^\infty e^{-x}\ket{2x} + \sum\limits_{x=0}^\infty e^{-x - 1/2}\ket{2x+1}) \\
            &= \sqrt{\frac{e-1}{e}} (\sum\limits_{x=0}^\infty e^{-x}\ket{2x} + e^{-1/2}\sum\limits_{x=0}^\infty e^{-x}\ket{2x+1}) \\
    \end{align*}

    Then we see that the ratio of probabilities of even and odd numbers is:
    \begin{equation*}
        \frac{P(\text{even})}{P(\text{odd})} = \frac{1^2}{(e^{-1/2})^2} = e
    \end{equation*}

    Then from probability:
    \begin{align*}
        1 &= P(\text{all}) = P(\text{odd}) + P(\text{even}) \\
          &= P(\text{even})\frac{P(\text{odd})}{P(\text{even})} + P(\text{even}) \\
          &= P(\text{even})e^{-1} + P(\text{even}) \\
          &= P(\text{even})(1 + \frac{1}{e}) = 1
    \end{align*}

    Hence:
    \begin{align*}
        P(\text{even})(1 + \frac{1}{e}) = 1 \\
        P(\text{even}) = \frac{1}{(1 + \frac{1}{e})} \\
        P(\text{even}) = \frac{e}{(e + 1)}
    \end{align*}

    \subsection*{Part (b)}

    First let's note that $e^{-x} > 0$ for all $x \in \mathbb{R}$. \\
    Hence:
    \begin{align*}
        1 &= \\
        \sum\limits_{i = 0}^0 e^{-i^2} &< \\
        \sum\limits_{i = 0}^1 e^{-i^2} &< \\
        \sum\limits_{i = 0}^2 e^{-i^2} &< \\
        ... &< \\
        \sum\limits_{i = 0}^\infty e^{-i^2} &< \\
    \end{align*}

    Next, due to convexity of the function $\sqrt{x} < \sqrt{x + \epsilon}$, where $x \ge 0, \epsilon > 0$.

    Both of these properties will be used in derivation below.

    The probability of a perfect square is:
    \begin{align*}
        (\sum\limits_{i = 0}^\infty (\sqrt{\frac{e - 1}{e}}e^{-i^2/2})^2)^\frac{1}{2} &= \\
        (\frac{e - 1}{e}\sum\limits_{i = 0}^\infty e^{-i^2})^\frac{1}{2} &= \\
        \sqrt{\frac{e - 1}{e}} (\sum\limits_{i = 0}^\infty e^{-i^2})^\frac{1}{2} &\ge \\
        0.79 \cdot (\sum\limits_{i = 0}^\infty e^{-i^2})^\frac{1}{2} &= \\
        0.79 \cdot (e^0 + e^{-1} + e^{-4} + \sum\limits_{i = 0}^\infty e^{-i^2})^\frac{1}{2} &> \\
        0.79 \cdot (1 + 0.36 + 0.01 + \sum\limits_{i = 0}^\infty e^{-i^2})^\frac{1}{2} &> \\
        0.79 \cdot 1^\frac{1}{2} &= \\
        0.79
    \end{align*}

    Let us calculate:
    \begin{equation*}
        \frac{e - 1}{e} \frac{\sqrt{\pi}}{2} < 0.57
    \end{equation*}

    Hence, overall:
    \begin{equation*}
        (\sum\limits_{i = 0}^\infty (\sqrt{\frac{e - 1}{e}}e^{-i^2/2})^2)^\frac{1}{2} > 0.79 > 0.57 > \frac{e - 1}{e} \frac{\sqrt{\pi}}{2}
    \end{equation*}


\end{homeworkProblem}
\end{document}