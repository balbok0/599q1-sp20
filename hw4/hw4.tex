\documentclass{article}

\usepackage{fancyhdr}
\usepackage{extramarks}
\usepackage{amsmath}
\usepackage{amsthm}
\usepackage{amsfonts}
\usepackage{tikz}
\usepackage[plain]{algorithm}
\usepackage{algpseudocode}
\usepackage{qcircuit}

\usetikzlibrary{automata,positioning}

%
% Basic Document Settings
%

\topmargin=-0.45in
\evensidemargin=0in
\oddsidemargin=0in
\textwidth=6.5in
\textheight=9.0in
\headsep=0.25in

\linespread{1.1}

\pagestyle{fancy}
\lhead{\hmwkAuthorName}
\chead{}
\rhead{\hmwkClass}
\lfoot{\lastxmark}
\cfoot{\thepage}

\renewcommand\headrulewidth{0.4pt}
\renewcommand\footrulewidth{0.4pt}

\setlength\parindent{0pt}

%
% Create Problem Sections
%

\newcommand{\enterProblemHeader}[1]{
    \nobreak\extramarks{}{Problem \arabic{#1} continued on next page\ldots}\nobreak{}
    \nobreak\extramarks{Problem \arabic{#1} (continued)}{Problem \arabic{#1} continued on next page\ldots}\nobreak{}
}

\newcommand{\exitProblemHeader}[1]{
    \nobreak\extramarks{Problem \arabic{#1} (continued)}{Problem \arabic{#1} continued on next page\ldots}\nobreak{}
    \stepcounter{#1}
    \nobreak\extramarks{Problem \arabic{#1}}{}\nobreak{}
}

\setcounter{secnumdepth}{0}
\newcounter{partCounter}
\newcounter{homeworkProblemCounter}
\setcounter{homeworkProblemCounter}{1}
\nobreak\extramarks{Problem \arabic{homeworkProblemCounter}}{}\nobreak{}

%
% Homework Problem Environment
%
% This environment takes an optional argument. When given, it will adjust the
% problem counter. This is useful for when the problems given for your
% assignment aren't sequential. See the last 3 problems of this template for an
% example.
%
\newenvironment{homeworkProblem}[1][-1]{
    \ifnum#1>0
        \setcounter{homeworkProblemCounter}{#1}
    \fi
    \section{Problem \arabic{homeworkProblemCounter}}
    \setcounter{partCounter}{1}
    \enterProblemHeader{homeworkProblemCounter}
}{
    \exitProblemHeader{homeworkProblemCounter}
}

%
% Homework Details
%   - Title
%   - Due date
%   - Class
%   - Section/Time
%   - Instructor
%   - Author
%

\newcommand{\hmwkTitle}{Homework\ \#1}
\newcommand{\hmwkDueDate}{April 28, 2020}
\newcommand{\hmwkClass}{Intorduction to Quantum Computing}
\newcommand{\hmwkAuthorName}{Jakub Filipek}

%
% Title Page
%

\title{
    % \vspace{in}
    \textmd{\textbf{\hmwkClass:\ \hmwkTitle}}\\
    \normalsize\vspace{0.1in}\small{Due\ on\ \hmwkDueDate}\\
}

\author{\hmwkAuthorName}
\date{}

\renewcommand{\part}[1]{\textbf{\large Part \Alph{partCounter}}\stepcounter{partCounter}\\}

%
% Various Helper Commands
%

% Useful for algorithms
\newcommand{\alg}[1]{\textsc{\bfseries \footnotesize #1}}

% For derivatives
\newcommand{\deriv}[1]{\frac{\mathrm{d}}{\mathrm{d}x} (#1)}

% For partial derivatives
\newcommand{\pderiv}[2]{\frac{\partial}{\partial #1} (#2)}

% Integral dx
\newcommand{\dx}{\mathrm{d}x}

% Alias for the Solution section header
\newcommand{\solution}{\textbf{\large Solution}}

% Probability commands: Expectation, Variance, Covariance, Bias
\newcommand{\E}{\mathrm{E}}
\newcommand{\Var}{\mathrm{Var}}
\newcommand{\Cov}{\mathrm{Cov}}
\newcommand{\Bias}{\mathrm{Bias}}

\newcommand{\norm}[1]{\left\lVert#1\right\rVert}

\newcommand{\bra}[1]{\lstick#1|}
\newcommand{\ket}[1]{|#1\rangle}
\newcommand{\qbra}{\bra{\psi}}
\newcommand{\qket}{\ket{\psi}}


\newcommand{\qwxo}[2][-1]{\ar @{-} [#1,0]|*+<2pt,4pt>[Fo]{#2}}

\begin{document}

\maketitle

% \pagebreak

\begin{homeworkProblem}
    \subsection*{Part (a)}
    There are exactly $n$ Hadamard gates and $n^2$ $R_Z$ gates in a QFT circuit.\\
    I will assume that Hadamards are implemented in an errorless fashion (this is unrealistic, but the problem, does not specify bounds on Hadamard gate error).

    Lastly, let's note that when gates are applied the error scales linearly with number of gates applied.

    \begin{align*}
        \norm{F_{2^n} - \widetilde{F}_{2^n}(\theta)} &\le \norm{n^2R_Z(\theta) - n^2 \widetilde{R}_Z(\theta)} \\
        &= n^2 \norm{R_Z(\theta) - \widetilde{R}_Z(\theta)} \\
        \le n^2 \Delta
    \end{align*}

    Since $\Delta \in O(\frac{\epsilon}{n^2})$, we can choose set $\epsilon = \frac{1}{k} n^2 \Delta$, where $k > 1$ (this is stronger that standard big-$O$ requirement).

    This means that we can choose $\epsilon$ to be tuned (by choosing $k$) to be exactly:
    \begin{align*}
        \norm{F_{2^n} - \widetilde{F}_{2^n}(\theta)} &= \epsilon \\
        &= \frac{1}{k} n^2 \Delta \\
        &\le n^2 \Delta
    \end{align*}

    Hence such implementation is possible.

    \subsection*{Part (b)}
    Firstly let us note that an Adder contains 2 QFT circuits, and an additional set of $n^2$ $R_Z$ gates.
    Hence, there is total of $3n^2$ $R_Z$ gates.

    \begin{align*}
        \norm{ADD - \widetilde{ADD}} &= \norm{3n^2R_Z - 3n^2\widetilde{R}_Z} \\
        &= 3n^2\norm{R_Z - \widetilde{R}_Z} \\
        &\le 3n^2 \Delta
    \end{align*}

    Let us set:
    $\epsilon = \frac{\Delta}{3n^2}$

    Then:
    \begin{equation*}
        \norm{ADD - \widetilde{ADD}} \le \epsilon
    \end{equation*}

    Note that $\max_\qket |(R_Z(\theta) - R_Z(\theta))\qket|$ can be at most $2$.
    This is because we are taking the difference between two unit vectors,
    so the maximum difference if they are anti-parallel to each other, in which case it is equal to $2$.
    Hence we can assume $\Delta$ is at most $2$.

    Hence:
    \begin{equation*}
        \epsilon \le \frac{2}{3n^2} < 1
    \end{equation*}

    Since $n^2 \ge 1$

    Aggregating above equations we get:
    \begin{equation*}
        \norm{ADD - \widetilde{ADD}} \le \epsilon < 1
    \end{equation*}

    \subsection*{Part (c)}
    Firstly let us note that worst case implementation of $\widetilde{ADD}$ will have error of $\epsilon$ in the state space.

    Now let us consider, what would be the worst case impact of error $\epsilon$ in the state space on the measured number.
    Since, both $\ket{x}$ and $\ket{y}$ have $n$ bits, their most significant bit represents power of $2^{n-1}$.

    This also represents the highest error in the number space, since if we make a mistake we would be $2^{n-1}$ away from true answer.
    Hence, let:
    \begin{align*}
        ADD\ket{x}\ket{y} \rightarrow \ket{x}\ket{a} \\
        \widetilde{ADD}\ket{x}\ket{y} \rightarrow \ket{x}\ket{\widetilde{a}}
    \end{align*}

    For discrete gradients the variance within a range can be maximized if there are only 2 observations possible, but they are maximally apart.
    Hence let:
    \begin{align*}
        \ket{a} = \begin{pmatrix}
            0 \\ \vdots \\ 1 \\ \vdots \\ 0
        \end{pmatrix} & &
        \ket{\widetilde{a}} = \begin{pmatrix}
            0 \\ \vdots \\ 1 - \frac{\epsilon}{2} \\ \vdots \\ \frac{\epsilon}{2} \\ \vdots \\ 0
        \end{pmatrix} \\
    \end{align*}

    Where $1 - \frac{\epsilon}{2}$ in $\ket{\widetilde{a}}$ is at the same index as the $1$ in $\ket{a}$,
    and $\frac{\epsilon}{2}$ is $2^{n - 1}$ indexes away from the previously mentioned elements (this can be either above or below them).

    Without loss of generality let us assume that $z$ is the index of $1 - \frac{\epsilon}{2}$, and that it's $\le 2^{n-1} - 1$.
    This way we can assume that index of $\frac{\epsilon}{2}$ is $z + 2^{n-1}$.
    In general, if z does not satisfy this bound we can just switch the computation below with minus sign, which doesn't change the overall bound.

    Hence, the overall variance will be:
    \begin{align*}
        Var(\widetilde{ADD}) &= \sum\limits_x P(x)f(x)^2 - (P(x)f(x))^2 \\
        &= ((1 - \frac{\epsilon}{2})z^2 - ((1 - \frac{\epsilon}{2})z)^2) + ((\frac{\epsilon}{2})(z + 2^{n - 1})^2 - (\frac{\epsilon}{2}(z + 2^{n - 1}))^2) \\
        &= z^2((1 - \frac{\epsilon}{2}) - (1 - \epsilon + \frac{\epsilon^2}{4})) + (z + 2^{n - 1})^2(\frac{\epsilon}{2} - \frac{\epsilon^2}{4}) \\
        &= z^2(\frac{\epsilon}{2} - \frac{\epsilon^2}{4}) + (z + 2^{n - 1})^2(\frac{\epsilon}{2} - \frac{\epsilon^2}{4}) \\
        &= (z^2 + (z + 2^{n - 1})^2)(\frac{\epsilon}{2} - \frac{\epsilon^2}{4}) \\
        &= (2z^2 + z2^n + 2^{2n - 2})(\frac{\epsilon}{2} - \frac{\epsilon^2}{4})
    \end{align*}

    Large values of $z$ maximize the above equation. However, since it is bounded by $2^{n-1} - 1$:
    \begin{align*}
        Var(\widetilde{ADD}) &= (2z^2 + z2^n + 2^{2n - 2})(\frac{\epsilon}{2} - \frac{\epsilon^2}{4}) \\
        &\le (2(2^{n-1} - 1)^2 + (2^{n-1} - 1)2^n + 2^{2n - 2})(\frac{\epsilon}{2} - \frac{\epsilon^2}{4}) \\
        &= (2(2^{2n-2} - 2^{n} + 1) + 2^{2n-1} - 2^n + 2^{2n - 2})(\frac{\epsilon}{2} - \frac{\epsilon^2}{4}) \\
        &= (2^{2n-1} - 2^{n + 1} + 2 + 2^{2n-1} - 2^n + 2^{2n - 2})(\frac{\epsilon}{2} - \frac{\epsilon^2}{4}) \\
        &= (2^{2n} + 2^{2n - 2} - 2^n - 2^{n + 1} + 2)(\frac{\epsilon}{2} - \frac{\epsilon^2}{4}) \\
        &\le (2^{2n} + 2^{2n - 2})(\frac{\epsilon}{2} - \frac{\epsilon^2}{4}) \\
        &\le (2 \cdot 2^{2n})(\frac{\epsilon}{2} - \frac{\epsilon^2}{4}) \\
        &= 2^{2n}(\epsilon - \frac{\epsilon^2}{2}) \\
        &\le 2^{2n} (\epsilon + \frac{\epsilon^2}{4}) \\
        &\le 2^{2n} 4\epsilon(1 + \frac{\epsilon}{4})
    \end{align*}

    \subsection*{Part (d)}
\end{homeworkProblem}
\end{document}