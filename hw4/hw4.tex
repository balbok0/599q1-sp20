\documentclass{article}

\usepackage{fancyhdr}
\usepackage{extramarks}
\usepackage{amsmath}
\usepackage{amsthm}
\usepackage{amsfonts}
\usepackage{tikz}
\usepackage[plain]{algorithm}
\usepackage{algpseudocode}
\usepackage{qcircuit}

\usetikzlibrary{automata,positioning}

%
% Basic Document Settings
%

\topmargin=-0.45in
\evensidemargin=0in
\oddsidemargin=0in
\textwidth=6.5in
\textheight=9.0in
\headsep=0.25in

\linespread{1.1}

\pagestyle{fancy}
\lhead{\hmwkAuthorName}
\chead{}
\rhead{\hmwkClass}
\lfoot{\lastxmark}
\cfoot{\thepage}

\renewcommand\headrulewidth{0.4pt}
\renewcommand\footrulewidth{0.4pt}

\setlength\parindent{0pt}

%
% Create Problem Sections
%

\newcommand{\enterProblemHeader}[1]{
    \nobreak\extramarks{}{Problem \arabic{#1} continued on next page\ldots}\nobreak{}
    \nobreak\extramarks{Problem \arabic{#1} (continued)}{Problem \arabic{#1} continued on next page\ldots}\nobreak{}
}

\newcommand{\exitProblemHeader}[1]{
    \nobreak\extramarks{Problem \arabic{#1} (continued)}{Problem \arabic{#1} continued on next page\ldots}\nobreak{}
    \stepcounter{#1}
    \nobreak\extramarks{Problem \arabic{#1}}{}\nobreak{}
}

\setcounter{secnumdepth}{0}
\newcounter{partCounter}
\newcounter{homeworkProblemCounter}
\setcounter{homeworkProblemCounter}{1}
\nobreak\extramarks{Problem \arabic{homeworkProblemCounter}}{}\nobreak{}

%
% Homework Problem Environment
%
% This environment takes an optional argument. When given, it will adjust the
% problem counter. This is useful for when the problems given for your
% assignment aren't sequential. See the last 3 problems of this template for an
% example.
%
\newenvironment{homeworkProblem}[1][-1]{
    \ifnum#1>0
        \setcounter{homeworkProblemCounter}{#1}
    \fi
    \section{Problem \arabic{homeworkProblemCounter}}
    \setcounter{partCounter}{1}
    \enterProblemHeader{homeworkProblemCounter}
}{
    \exitProblemHeader{homeworkProblemCounter}
}

%
% Homework Details
%   - Title
%   - Due date
%   - Class
%   - Section/Time
%   - Instructor
%   - Author
%

\newcommand{\hmwkTitle}{Homework\ \#1}
\newcommand{\hmwkDueDate}{April 28, 2020}
\newcommand{\hmwkClass}{Intorduction to Quantum Computing}
\newcommand{\hmwkAuthorName}{Jakub Filipek}

%
% Title Page
%

\title{
    % \vspace{in}
    \textmd{\textbf{\hmwkClass:\ \hmwkTitle}}\\
    \normalsize\vspace{0.1in}\small{Due\ on\ \hmwkDueDate}\\
}

\author{\hmwkAuthorName}
\date{}

\renewcommand{\part}[1]{\textbf{\large Part \Alph{partCounter}}\stepcounter{partCounter}\\}

%
% Various Helper Commands
%

% Useful for algorithms
\newcommand{\alg}[1]{\textsc{\bfseries \footnotesize #1}}

% For derivatives
\newcommand{\deriv}[1]{\frac{\mathrm{d}}{\mathrm{d}x} (#1)}

% For partial derivatives
\newcommand{\pderiv}[2]{\frac{\partial}{\partial #1} (#2)}

% Integral dx
\newcommand{\dx}{\mathrm{d}x}

% Alias for the Solution section header
\newcommand{\solution}{\textbf{\large Solution}}

% Probability commands: Expectation, Variance, Covariance, Bias
\newcommand{\E}{\mathrm{E}}
\newcommand{\Var}{\mathrm{Var}}
\newcommand{\Cov}{\mathrm{Cov}}
\newcommand{\Bias}{\mathrm{Bias}}

\newcommand{\norm}[1]{\left\lVert#1\right\rVert}

\newcommand{\bra}[1]{\lstick#1|}
\newcommand{\ket}[1]{|#1\rangle}
\newcommand{\qbra}{\bra{\psi}}
\newcommand{\qket}{\ket{\psi}}


\newcommand{\qwxo}[2][-1]{\ar @{-} [#1,0]|*+<2pt,4pt>[Fo]{#2}}

\begin{document}

\maketitle

% \pagebreak

\begin{homeworkProblem}
    \subsection*{Part (a)}
    There are exactly $n$ Hadamard gates and $n^2$ $R_Z$ gates in a QFT circuit.\\
    I will assume that Hadamards are implemented in an errorless fashion (this is unrealistic, but the problem, does not specify bounds on Hadamard gate error).

    \begin{align*}
        \norm{F_{2^n} - \widetilde{F}_{2^n}(\theta)} &= \norm{n^2R_Z(\theta) - n^2 \widetilde{R}_Z(\theta)} \\
        &= n^2 \norm{R_Z(\theta) - \widetilde{R}_Z(\theta)} \\
        \le n^2 \Delta
    \end{align*}

    Since $\Delta \in O(\frac{\epsilon}{n^2})$, we can choose set $\epsilon = \frac{1}{k} n^2 \Delta$, where $k > 1$ (this is stronger that standard big-$O$ requirement).

    This means that we can choose $\epsilon$ to be tuned (by choosing $k$) to be exactly:
    \begin{align*}
        \norm{F_{2^n} - \widetilde{F}_{2^n}(\theta)} &= \epsilon \\
        &= \frac{1}{k} n^2 \Delta \\
        &\le n^2 \Delta
    \end{align*}

    Hence such implementation is possible.

    \subsection*{Part (b)}
    Firstly let us note that an Adder contains 2 QFT circuits, and an additional set of $n^2$ $R_Z$ gates.
    Hence, there is total of $3n^2$ $R_Z$ gates.

    \begin{align*}
        \norm{ADD - \widetilde{ADD}} &= \norm{3n^2R_Z - 3n^2\widetilde{R}_Z} \\
        &= 3n^2\norm{R_Z - \widetilde{R}_Z} \\
        &\le 3n^2 \Delta
    \end{align*}

    Let us set:
    $\epsilon = \frac{\Delta}{3n^2}$

    Then:
    \begin{equation*}
        \norm{ADD - \widetilde{ADD}} \le \epsilon
    \end{equation*}

    Note that $\max_\qket |(R_Z(\theta) - R_Z(\theta))\qket|$ can be at most $2$.
    This is because we are taking the difference between two unit vectors,
    so the maximum difference if they are anti-parallel to each other, in which case it is equal to $2$.
    Hence we can assume $\Delta$ is at most $2$.

    Hence:
    \begin{equation*}
        \epsilon \le \frac{2}{3n^2} < 1
    \end{equation*}

    Since $n^2 \ge 1$

    Aggregating above equations we get:
    \begin{equation*}
        \norm{ADD - \widetilde{ADD}} \le \epsilon < 1
    \end{equation*}

    \subsection*{Part (c)}

    
\end{homeworkProblem}
\end{document}