\documentclass{article}

\usepackage{fancyhdr}
\usepackage{extramarks}
\usepackage{amsmath}
\usepackage{amsthm}
\usepackage{amsfonts}
\usepackage{tikz}
\usepackage[plain]{algorithm}
\usepackage{algpseudocode}
\usepackage{qcircuit}
\usepackage{hyperref}
\hypersetup{
    colorlinks=true,
    linkcolor=blue,
    filecolor=magenta,
    urlcolor=purple,
}

\urlstyle{same}

\usetikzlibrary{automata,positioning}

%
% Basic Document Settings
%

\topmargin=-0.45in
\evensidemargin=0in
\oddsidemargin=0in
\textwidth=6.5in
\textheight=9.0in
\headsep=0.25in

\linespread{1.1}

\pagestyle{fancy}
\lhead{\hmwkAuthorName}
\chead{}
\rhead{\hmwkClass}
\lfoot{\lastxmark}
\cfoot{\thepage}

\renewcommand\headrulewidth{0.4pt}
\renewcommand\footrulewidth{0.4pt}

\setlength\parindent{0pt}

%
% Create Problem Sections
%

\newcommand{\enterProblemHeader}[1]{
    \nobreak\extramarks{}{Problem \arabic{#1} continued on next page\ldots}\nobreak{}
    \nobreak\extramarks{Problem \arabic{#1} (continued)}{Problem \arabic{#1} continued on next page\ldots}\nobreak{}
}

\newcommand{\exitProblemHeader}[1]{
    \nobreak\extramarks{Problem \arabic{#1} (continued)}{Problem \arabic{#1} continued on next page\ldots}\nobreak{}
    \stepcounter{#1}
    \nobreak\extramarks{Problem \arabic{#1}}{}\nobreak{}
}

\setcounter{secnumdepth}{0}
\newcounter{partCounter}
\newcounter{homeworkProblemCounter}
\setcounter{homeworkProblemCounter}{1}
\nobreak\extramarks{Problem \arabic{homeworkProblemCounter}}{}\nobreak{}

%
% Homework Problem Environment
%
% This environment takes an optional argument. When given, it will adjust the
% problem counter. This is useful for when the problems given for your
% assignment aren't sequential. See the last 3 problems of this template for an
% example.
%
\newenvironment{homeworkProblem}[1][-1]{
    \ifnum#1>0
        \setcounter{homeworkProblemCounter}{#1}
    \fi
    \section{Problem \arabic{homeworkProblemCounter}}
    \setcounter{partCounter}{1}
    \enterProblemHeader{homeworkProblemCounter}
}{
    \exitProblemHeader{homeworkProblemCounter}
}

%
% Homework Details
%   - Title
%   - Due date
%   - Class
%   - Section/Time
%   - Instructor
%   - Author
%

\newcommand{\hmwkTitle}{Homework\ \#5}
\newcommand{\hmwkDueDate}{May 15th 2020}
\newcommand{\hmwkClass}{Intorduction to Quantum Computing}
\newcommand{\hmwkAuthorName}{Jakub Filipek}

%
% Title Page
%

\title{
    % \vspace{in}
    \textmd{\textbf{\hmwkClass:\ \hmwkTitle}}\\
    \normalsize\vspace{0.1in}\small{Due\ on\ \hmwkDueDate}\\
}

\author{\hmwkAuthorName}
\date{}

\renewcommand{\part}[1]{\textbf{\large Part \Alph{partCounter}}\stepcounter{partCounter}\\}

%
% Various Helper Commands
%

% Useful for algorithms
\newcommand{\alg}[1]{\textsc{\bfseries \footnotesize #1}}

% For derivatives
\newcommand{\deriv}[1]{\frac{\mathrm{d}}{\mathrm{d}x} (#1)}

% For partial derivatives
\newcommand{\pderiv}[2]{\frac{\partial}{\partial #1} (#2)}

% Integral dx
\newcommand{\dx}{\mathrm{d}x}

% Alias for the Solution section header
\newcommand{\solution}{\textbf{\large Solution}}

% Probability commands: Expectation, Variance, Covariance, Bias
\newcommand{\E}{\mathrm{E}}
\newcommand{\Var}{\mathrm{Var}}
\newcommand{\Cov}{\mathrm{Cov}}
\newcommand{\Bias}{\mathrm{Bias}}

\newcommand{\norm}[1]{\left\lVert#1\right\rVert}

\newcommand{\bra}[1]{\lstick#1|}
\newcommand{\ket}[1]{|#1\rangle}
\newcommand{\qbra}{\bra{\psi}}
\newcommand{\qket}{\ket{\psi}}


\newcommand{\qwxo}[2][-1]{\ar @{-} [#1,0]|*+<2pt,4pt>[Fo]{#2}}

\begin{document}

\maketitle

% \pagebreak

\begin{homeworkProblem}
    \subsection*{Part (a)}

    \subsection*{Part (b)}
\end{homeworkProblem}

\vspace{2cm}
\begin{homeworkProblem}
    \subsection*{Part (a)}
    Let $\qket = a\ket{0} + b\ket{1}$

    Then:
    \begin{align*}
        a\ket{00} + b\ket{01} &\xrightarrow{e^{-iX\theta}, I} \\
        a\cos(\theta)\ket{00} - a i \sin(\theta)\ket{10} + b \cos(\theta) \ket{01} - b i \sin(\theta)\ket{11} &\xrightarrow{C(-iX)} \\
        a\cos(\theta)\ket{00} + a \sin(\theta)\ket{11} + b \cos(\theta) \ket{01} + b \sin(\theta)\ket{10} &\xrightarrow{e^{iX\theta}, I} \\
        \text{ } \\
        a\cos^2(\theta)\ket{00} + ai\cos(\theta)\sin(\theta)\ket{10} + \\
        ai\sin^2(\theta)\ket{01} + a\cos(\theta)\sin(\theta)\ket{11} + \\
        b\cos^2(\theta)\ket{01} + bi\sin(\theta)\cos(\theta)\ket{11} + \\
        bi\sin^2(\theta)\ket{00} + b\cos(\theta)\sin(\theta)\ket{10} &= \\
        \text{ } \\
        (a\cos^2(\theta) + bi\sin^2(\theta))\ket{00} + \\
        (ai\sin^2(\theta) + b\cos^2(\theta))\ket{01} + \\
        (ai\cos(\theta)\sin(\theta) + b\cos(\theta)\sin(\theta))\ket{10} + \\
        (a\cos(\theta)\sin(\theta) + bi\sin(\theta)\cos(\theta))\ket{11}
    \end{align*}

    Now let us analyze what happens when we measure $\ket{1}$ in the first qubit.
    Then the we can see that the probability of the $\ket{0}$ and $\ket{1}$ states respectively in the second register are
    (up to a renormalization factor, which is the same in both cases):
    \begin{align*}
        \ket{0} &: (a^2\cos^2(\theta)\sin^2(\theta) + b^2\cos^2(\theta)\sin^2(\theta)) \\
        \ket{1} &: (a^2\cos^2(\theta)\sin^2(\theta) + b^2\cos^2(\theta)\sin^2(\theta))
    \end{align*}

    We can see that these are exactly the same, and hence the amplitudes have to be $\frac{1}{\sqrt{2}}$. Keeping the $i$ one in place,
    and considering that each term with $a$ came from $\ket{0}$ and $b$ came from $\ket{1}$ we get:
    \begin{equation*}
        \frac{1}{\sqrt{2}}
        \begin{pmatrix}
            i & 1 \\
            1 & i \\
        \end{pmatrix}
    \end{equation*},
    which up to a global phase ($-i$) implements:
    \begin{equation*}
        e^{-iX\frac{\pi}{4}}
    \end{equation*}


    For the other branch, where second register is $\ket{0}$ let us first compute the normalization factor:
    \begin{align*}
        \sqrt{|a\cos^2(\theta) + bi\sin^2(\theta)|^2 + |ai\sin^2(\theta) + b\cos^2(\theta)|^2} &= \\
        \sqrt{(a^2\cos^4(\theta) + b^2\sin^4(\theta)) + (a^2\sin^4(\theta) + b^2\cos^4(\theta))} &= \\
        \sqrt{(a^2 + b^2)(\sin^4(\theta) + \cos^4(\theta))}
    \end{align*}

    Then the probability of measuring states $\ket{0}$ and $\ket{1}$ are:
    \begin{align*}
        \ket{0} &: \frac{a\cos^2(\theta) + ib\sin^2(\theta)}{\sqrt{(a^2 + b^2)(\sin^4(\theta) + \cos^4(\theta))}} \\
        \ket{1} &: \frac{ia\sin^2(\theta) + b\cos^2(\theta)}{\sqrt{(a^2 + b^2)(\sin^4(\theta) + \cos^4(\theta))}}
    \end{align*}

    Note that $a^2 + b^2 = 1$, since we assume that $\qket$ is properly normalized state.\\
    Using the previously mentioned fact about how $a$ and $b$ indicate which initial state
    contributed that part of equation, and hence $a, b$ are not part of the unitary operation, we get:
    \begin{align*}
        \frac{1}{\sqrt{\sin^4(\theta) + \cos^4(\theta)}}
        \begin{pmatrix}
            \cos^2(\theta) & i\sin^2(\theta) \\
            i\sin^2(\theta) & \cos^2(\theta) \\
        \end{pmatrix}
    \end{align*}

    Note that ():
    \begin{align*}
        \frac{\cos^2(\theta)}{\sqrt{\sin^4(\theta) + \cos^4(\theta)}} &= \\
        \frac{1}{\sqrt{\frac{\sin^4(\theta)}{\cos^4(\theta)} + 1}} &= \\
        \frac{1}{\sqrt{\tan^4(\theta) + 1}} &= \\
        \cos(\arctan(\tan^2(\theta))) \\
        \text{ } \\
        \frac{\sin^2(\theta)}{\sqrt{\sin^4(\theta) + \cos^4(\theta)}} &= \\
        \frac{1}{\sqrt{\frac{\cos^4(\theta)}{\sin^4(\theta)} + 1}} &= \\
        \frac{1}{\sqrt{\cot^4(\theta) + 1}} &= \\
        \sin(\arctan(\tan^2(\theta)))
    \end{align*}

    Hence we can write the above operation as:
    \begin{equation*}
        \begin{pmatrix}
            \cos(\arctan(\tan^2(\theta))) & i\sin(\arctan(\tan^2(\theta))) \\
            i\sin(\arctan(\tan^2(\theta))) & \cos(\arctan(\tan^2(\theta))) \\
        \end{pmatrix} =
        e^{i\arctan(\tan^2(\theta))X}
    \end{equation*}

    Hence, we get what the problem asked us for.\\
    (Up to a minus sign. I think I might have swapped the $e^{-i\theta X}$ with $e^{i\theta X}$)

    \subsection*{Part (b)}

\end{homeworkProblem}
\end{document}