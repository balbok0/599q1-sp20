\documentclass{article}

\usepackage{fancyhdr}
\usepackage{extramarks}
\usepackage{amsmath}
\usepackage{amsthm}
\usepackage{amsfonts}
\usepackage{tikz}
\usepackage[plain]{algorithm}
\usepackage{algpseudocode}
\usepackage{qcircuit}
\usepackage{hyperref}
\hypersetup{
    colorlinks=true,
    linkcolor=blue,
    filecolor=magenta,
    urlcolor=purple,
}

\urlstyle{same}

\usetikzlibrary{automata,positioning}

%
% Basic Document Settings
%

\topmargin=-0.45in
\evensidemargin=0in
\oddsidemargin=0in
\textwidth=6.5in
\textheight=9.0in
\headsep=0.25in

\linespread{1.1}

\pagestyle{fancy}
\lhead{\hmwkAuthorName}
\chead{}
\rhead{\hmwkClass}
\lfoot{\lastxmark}
\cfoot{\thepage}

\renewcommand\headrulewidth{0.4pt}
\renewcommand\footrulewidth{0.4pt}

\setlength\parindent{0pt}

%
% Create Problem Sections
%

\newcommand{\enterProblemHeader}[1]{
    \nobreak\extramarks{}{Problem \arabic{#1} continued on next page\ldots}\nobreak{}
    \nobreak\extramarks{Problem \arabic{#1} (continued)}{Problem \arabic{#1} continued on next page\ldots}\nobreak{}
}

\newcommand{\exitProblemHeader}[1]{
    \nobreak\extramarks{Problem \arabic{#1} (continued)}{Problem \arabic{#1} continued on next page\ldots}\nobreak{}
    \stepcounter{#1}
    \nobreak\extramarks{Problem \arabic{#1}}{}\nobreak{}
}

\setcounter{secnumdepth}{0}
\newcounter{partCounter}
\newcounter{homeworkProblemCounter}
\setcounter{homeworkProblemCounter}{1}
\nobreak\extramarks{Problem \arabic{homeworkProblemCounter}}{}\nobreak{}

%
% Homework Problem Environment
%
% This environment takes an optional argument. When given, it will adjust the
% problem counter. This is useful for when the problems given for your
% assignment aren't sequential. See the last 3 problems of this template for an
% example.
%
\newenvironment{homeworkProblem}[1][-1]{
    \ifnum#1>0
        \setcounter{homeworkProblemCounter}{#1}
    \fi
    \section{Problem \arabic{homeworkProblemCounter}}
    \setcounter{partCounter}{1}
    \enterProblemHeader{homeworkProblemCounter}
}{
    \exitProblemHeader{homeworkProblemCounter}
}

%
% Homework Details
%   - Title
%   - Due date
%   - Class
%   - Section/Time
%   - Instructor
%   - Author
%

\newcommand{\hmwkTitle}{Homework\ \#5}
\newcommand{\hmwkDueDate}{May 15th 2020}
\newcommand{\hmwkClass}{Intorduction to Quantum Computing}
\newcommand{\hmwkAuthorName}{Jakub Filipek}

%
% Title Page
%

\title{
    % \vspace{in}
    \textmd{\textbf{\hmwkClass:\ \hmwkTitle}}\\
    \normalsize\vspace{0.1in}\small{Due\ on\ \hmwkDueDate}\\
}

\author{\hmwkAuthorName}
\date{}

\renewcommand{\part}[1]{\textbf{\large Part \Alph{partCounter}}\stepcounter{partCounter}\\}

%
% Various Helper Commands
%

% Useful for algorithms
\newcommand{\alg}[1]{\textsc{\bfseries \footnotesize #1}}

% For derivatives
\newcommand{\deriv}[1]{\frac{\mathrm{d}}{\mathrm{d}x} (#1)}

% For partial derivatives
\newcommand{\pderiv}[2]{\frac{\partial}{\partial #1} (#2)}

% Integral dx
\newcommand{\dx}{\mathrm{d}x}

% Alias for the Solution section header
\newcommand{\solution}{\textbf{\large Solution}}

% Probability commands: Expectation, Variance, Covariance, Bias
\newcommand{\E}{\mathrm{E}}
\newcommand{\Var}{\mathrm{Var}}
\newcommand{\Cov}{\mathrm{Cov}}
\newcommand{\Bias}{\mathrm{Bias}}

\newcommand{\norm}[1]{\left\lVert#1\right\rVert}

\newcommand{\bra}[1]{\langle#1|}
\newcommand{\ket}[1]{|#1\rangle}
\newcommand{\qbra}{\bra{\psi}}
\newcommand{\qket}{\ket{\psi}}


\newcommand{\qwxo}[2][-1]{\ar @{-} [#1,0]|*+<2pt,4pt>[Fo]{#2}}

\begin{document}

\maketitle

% \pagebreak

\begin{homeworkProblem}
    \subsection*{Part (a)}


    Any one state error on $a\ket{0}_L + b\ket{1}_L$ (i.e. after the CNOTs) would result in something similar to this:
    \begin{align*}
        a\ket{0}_L + b\ket{1}_L &\xrightarrow{I \otimes e^{-iX\theta} \otimes I} \\
        a\ket{0}(\cos(\theta)\ket{0} - i\sin(\theta)\ket{1})\ket{0} + \\
        b\ket{1}(- i\sin(\theta)\ket{0} + \cos(\theta)\ket{1})\ket{1} = \\
        \text{ } \\
        a\cos(\theta)\ket{000} - ai\sin(\theta)\ket{010} - \\
        bi\sin(\theta)\ket{101} + b\cos(\theta)\ket{111}\\
    \end{align*}

    \[\Qcircuit @C=1em @R=1em {
        & \qw      & \gate{Z} \qwx[1] & \gate{Z} \qwx[2] & \qw              & \qw      & \qw    & \qw \\
        & \qw      & \gate{Z}         & \qw              & \gate{Z} \qwx[1] & \qw      & \qw    & \qw \\
        & \qw      & \qw              & \gate{Z}         & \gate{Z}         & \qw      & \qw    & \qw \\
        & \gate{H} & \ctrl{-2}        & \qw              & \qw              & \gate{H} & \meter & \qw \\
        & \gate{H} & \qw              & \ctrl{-2}        & \qw              & \gate{H} & \meter & \qw \\
        & \gate{H} & \qw              & \qw              & \ctrl{-3}        & \gate{H} & \meter & \qw \\
    }\]

    Let us for example that the first stabilizer code (one with $Z$'s on two first registers).

    Let's also ignore qubit in the third register, since no gates are applied to it by the first stabilizer.
    Then (starting after first Hadamard, and omitting normalization factors):

    \begin{align*}
        (\ket{0} + \ket{1})(a\cos(\theta)\ket{00} - ai\sin(\theta)\ket{01} -bi\sin(\theta)\ket{10} + b\cos(\theta)\ket{11}) &\xrightarrow{C(ZZ)} \\
        \text{ } \\
        \ket{0}(a\cos(\theta)\ket{00} - ai\sin(\theta)\ket{01} - bi\sin(\theta)\ket{10} + b\cos(\theta)\ket{11}) &+ \\
        \ket{1}(a\cos(\theta)\ket{00} + ai\sin(\theta)\ket{01} + bi\sin(\theta)\ket{10} + b\cos(\theta)\ket{11}) &\xrightarrow{IIH} \\
        \text{ } \\
        \ket{0}(a\cos(\theta)\ket{00} - ai\sin(\theta)\ket{01} - bi\sin(\theta)\ket{10} + b\cos(\theta)\ket{11}) &+ \\
        \ket{1}(a\cos(\theta)\ket{00} - ai\sin(\theta)\ket{01} - bi\sin(\theta)\ket{10} + b\cos(\theta)\ket{11}) &+ \\
        \ket{0}(a\cos(\theta)\ket{00} + ai\sin(\theta)\ket{01} + bi\sin(\theta)\ket{10} + b\cos(\theta)\ket{11}) &- \\
        \ket{1}(a\cos(\theta)\ket{00} + ai\sin(\theta)\ket{01} + bi\sin(\theta)\ket{10} + b\cos(\theta)\ket{11}) &= \\
        \text{ } \\
        2\ket{0}(a\cos(\theta)\ket{00} + b\cos(\theta)\ket{11}) - 2\ket{1}(ai\sin(\theta)\ket{01} + bi\sin(\theta)\ket{10})
    \end{align*}

    Now assume that we measured $\ket{0}$ in the stabilizer. This means that the overall state (after proper normalization)
    will collapse to $a\ket{000} + b\ket{111}$, which implies that there are no errors.

    On the other hand if we measure $\ket{1}$ in the stabilizer, the state will collapse to $a\ket{01} + b\ket{10}$ (up to a global phase).
    Hence, $\ket{1}$ implies that there are errors.
    Then we do the same trick as shown in class, to correct the first state.

    \paragraph*{}
    However, there are still two things to show. What happens when there is no error for a given stabilizer (i.e. the state is in $a\ket{00} + b\ket{11}$).
    Then we get:
    \begin{align*}
        (\ket{0} + \ket{1})(a\ket{00} + \ket{11}) &\xrightarrow{CZZ} \\
        (\ket{0} + \ket{1})(a\ket{00} + \ket{11}) &\xrightarrow{IIH} \\
        \ket{0}(a\ket{00} + \ket{11})
    \end{align*}
    Which aligns with the fact that $\ket{0}$ in the stabilizer implies no error in the state.

    \paragraph*{}
    Lastly, we need to argue that stabilizers are entangled. In other words we cannot measure odd number of $\ket{1}$ in stabilizers.
    This can seen by the fact that values of stabilizers ($\ket{0} vs. \ket{1}$) are entangled with the same qubits (not exactly the same pairs, but the same qubits in these different pairs).
    Hence the stabilizers themselves are entangled.

    \subsection*{Part (b)}
    The 5 qubit Phase flip code can be just an extended version of 3 bit presented in class:
    \[
        \Qcircuit @C=1em @R=1em {
            \lstick{a\ket{0} + b\ket{1}} & \ctrl{1}       & \gate{H} & \qw \\
            \lstick{\ket{0}}             & \targ          & \gate{H} & \qw \\
            \lstick{\ket{0}}             & \targ \qwx[-1] & \gate{H} & \qw \\
            \lstick{\ket{0}}             & \targ \qwx[-1] & \gate{H} & \qw \\
            \lstick{\ket{0}}             & \targ \qwx[-1] & \gate{H} & \qw \\
        }
    \]

    This results in a state:
    $a\ket{+++++} + b\ket{-----}$

    \paragraph*{}
    Projectors:
    \begin{itemize}
        \item With 0 errors: $\hat{P}_0 = \ket{+++++}\bra{+++++} + \ket{-----}\bra{-----}$
        \item With 1 error: $\hat{P}_{1i} = \bigotimes\limits_{j=0}^4 I(1 - \delta_{ij}) + Z\delta_{ij}$,
            where $i \in [0, 4]$
        \item With 2 errors: $\hat{P}_{2ij} = \bigotimes\limits_{k=0}^4 I(1 - \delta_{ik}\delta{jk}) + Z(\delta_{ik} + \delta_{jk})$,
            where $i, j \in [0, 4]$ and $i \ne j$
    \end{itemize}

    \subsection*{Part (c)}
    It was mentioned in the lecture that the difference between phase-flip and bit-flip can be done by exchanging $X \leftrightarrow Z$.

    More formally this happens because for states:
    \begin{align*}
        X\ket{+} = \ket{+} \\
        X\ket{-} = -\ket{-} \\
    \end{align*}
    which implies that $\ket{+}, \ket{-}$ are eigenvectors of $X$, with $\lambda = 1, -1$, respectively.

    Hence, in if we have an even number of controlled $X$ gates,
    then (in the errorless) case we should get $\lambda = 1^4 = (-1)^4 = 1$.
    However, in case of error we can get $\lambda = -1$.

    Hence let us introduce codes:
    \begin{itemize}
        \item $S_1 = XXXXI$
        \item $S_2 = XXXIX$
        \item $S_3 = XXIXX$
        \item $S_4 = XIXXX$
        \item $S_5 = IXXXX$
    \end{itemize}
    Note that $S_i$ checks if there exists an error in any but the $i^{th}$ register.

    For one qubit errors we will have to correct a qubit corresponding to $i^{th}$ code,
    iff that code has $\lambda = 1$ (there will be only one such register).
    In other words, four of the stabilizers will have $\lambda = -1$,
    and one will have $\lambda = 1$.

    For two qubit errors we will have to correct a qubits corresponding to $i^{th}$ codes,
    iff that code has $\lambda = -1$ (there will be two such registers).
    In other words, three of the stabilizers will have $\lambda = 1$,
    and two will have $\lambda = -1$.


    Then in general (for one or two qubits) we can take majority sign of the stabilizers,
    and then correct (by applying $Z$ gate) all registers corresponding to stabilizers with
    minority sign.

    \subsection*{Part (d)}
    \[\Qcircuit @C=1em @R=1em {
        & \qw      & \qw              & \gate{X} \qwx[2] & \gate{X} \qwx[1] & \gate{X} \qwx[1] & \gate{X} \qwx[1] & \qw      & \qw    & \qw \\
        & \qw      & \gate{X} \qwx[1] & \qw              & \gate{X} \qwx[2] & \gate{X} \qwx[1] & \gate{X} \qwx[1] & \qw      & \qw    & \qw \\
        & \qw      & \gate{X} \qwx[1] & \gate{X} \qwx[1] & \qw              & \gate{X} \qwx[2] & \gate{X} \qwx[1] & \qw      & \qw    & \qw \\
        & \qw      & \gate{X} \qwx[1] & \gate{X} \qwx[1] & \gate{X} \qwx[1] & \qw              & \gate{X}         & \qw      & \qw    & \qw \\
        & \qw      & \gate{X}         & \gate{X}         & \gate{X}         & \gate{X}         & \qw              & \qw      & \qw    & \qw \\
        & \gate{H} & \ctrl{-1}        & \qw              & \qw              & \qw              & \qw              & \gate{H} & \meter & \cw \\
        & \gate{H} & \qw              & \ctrl{-2}        & \qw              & \qw              & \qw              & \gate{H} & \meter & \cw \\
        & \gate{H} & \qw              & \qw              & \ctrl{-3}        & \qw              & \qw              & \gate{H} & \meter & \cw \\
        & \gate{H} & \qw              & \qw              & \qw              & \ctrl{-4}        & \qw              & \gate{H} & \meter & \cw \\
        & \gate{H} & \qw              & \qw              & \qw              & \qw              & \ctrl{-6}        & \gate{H} & \meter & \cw \\
        }\]

        After the measurement the previously described procedure (at the end of part d) of applying a $Z$ gate to registers
        corresponding to minority sign stabilizers should be done.
\end{homeworkProblem}

\vspace{2cm}
\begin{homeworkProblem}
    \subsection*{Part (a)}
    Let $\qket = a\ket{0} + b\ket{1}$

    Then:
    \begin{align*}
        a\ket{00} + b\ket{01} &\xrightarrow{e^{-iX\theta}, I} \\
        a\cos(\theta)\ket{00} - a i \sin(\theta)\ket{10} + b \cos(\theta) \ket{01} - b i \sin(\theta)\ket{11} &\xrightarrow{C(-iX)} \\
        a\cos(\theta)\ket{00} - a \sin(\theta)\ket{11} + b \cos(\theta) \ket{01} - b \sin(\theta)\ket{10} &\xrightarrow{e^{iX\theta}, I} \\
        \text{ } \\
        a\cos^2(\theta)\ket{00} + ai\cos(\theta)\sin(\theta)\ket{10} + \\
        -ai\sin^2(\theta)\ket{01} - a\cos(\theta)\sin(\theta)\ket{11} + \\
        b\cos^2(\theta)\ket{01} + bi\sin(\theta)\cos(\theta)\ket{11} + \\
        -bi\sin^2(\theta)\ket{00} - b\cos(\theta)\sin(\theta)\ket{10} &= \\
        \text{ } \\
        (a\cos^2(\theta) - bi\sin^2(\theta))\ket{00} + \\
        (-ai\sin^2(\theta) + b\cos^2(\theta))\ket{01} + \\
        (ai\cos(\theta)\sin(\theta) - b\cos(\theta)\sin(\theta))\ket{10} + \\
        (-a\cos(\theta)\sin(\theta) + bi\sin(\theta)\cos(\theta))\ket{11}
    \end{align*}

    Now let us analyze what happens when we measure $\ket{1}$ in the first qubit.
    Then the we can see that the probability of the $\ket{0}$ and $\ket{1}$ states respectively in the second register are
    (up to a renormalization factor, which is the same in both cases):
    \begin{align*}
        \ket{0} &: (a^2\cos^2(\theta)\sin^2(\theta) + b^2\cos^2(\theta)\sin^2(\theta)) \\
        \ket{1} &: (a^2\cos^2(\theta)\sin^2(\theta) + b^2\cos^2(\theta)\sin^2(\theta))
    \end{align*}

    We can see that these are exactly the same, and hence the amplitudes have to be $\frac{1}{\sqrt{2}}$. Keeping the $i$ one in place,
    and considering that each term with $a$ came from $\ket{0}$ and $b$ came from $\ket{1}$ we get:
    \begin{equation*}
        \frac{1}{\sqrt{2}}
        \begin{pmatrix}
            i & -1 \\
            -1 & i \\
        \end{pmatrix}
    \end{equation*},
    which up to a global phase ($-i$) implements:
    \begin{equation*}
        e^{iX\frac{\pi}{4}}
    \end{equation*}


    For the other branch, where second register is $\ket{0}$ let us first compute the normalization factor:
    \begin{align*}
        \sqrt{|a\cos^2(\theta) - bi\sin^2(\theta)|^2 + |-ai\sin^2(\theta) + b\cos^2(\theta)|^2} &= \\
        \sqrt{(a^2\cos^4(\theta) + b^2\sin^4(\theta)) + (a^2\sin^4(\theta) + b^2\cos^4(\theta))} &= \\
        \sqrt{(a^2 + b^2)(\sin^4(\theta) + \cos^4(\theta))}
    \end{align*}

    Then the wave functions of states $\ket{0}$ and $\ket{1}$ in second register are are:
    \begin{align*}
        \ket{0} &: \frac{a\cos^2(\theta) - ib\sin^2(\theta)}{\sqrt{(a^2 + b^2)(\sin^4(\theta) + \cos^4(\theta))}} \\
        \ket{1} &: \frac{-ia\sin^2(\theta) + b\cos^2(\theta)}{\sqrt{(a^2 + b^2)(\sin^4(\theta) + \cos^4(\theta))}}
    \end{align*}

    Note that $a^2 + b^2 = 1$, since we assume that $\qket$ is properly normalized state.\\
    Using the previously mentioned fact about how $a$ and $b$ indicate which initial state
    contributed that part of equation, and hence $a, b$ are not part of the unitary operation, we get:
    \begin{align*}
        \frac{1}{\sqrt{\sin^4(\theta) + \cos^4(\theta)}}
        \begin{pmatrix}
            \cos^2(\theta) & -i\sin^2(\theta) \\
            -i\sin^2(\theta) & \cos^2(\theta) \\
        \end{pmatrix}
    \end{align*}

    Note that ():
    \begin{align*}
        \frac{\cos^2(\theta)}{\sqrt{\sin^4(\theta) + \cos^4(\theta)}} &= \\
        \frac{1}{\sqrt{\frac{\sin^4(\theta)}{\cos^4(\theta)} + 1}} &= \\
        \frac{1}{\sqrt{\tan^4(\theta) + 1}} &= \\
        \cos(\arctan(\tan^2(\theta))) \\
        \text{ } \\
        \frac{\sin^2(\theta)}{\sqrt{\sin^4(\theta) + \cos^4(\theta)}} &= \\
        \frac{1}{\sqrt{\frac{\cos^4(\theta)}{\sin^4(\theta)} + 1}} &= \\
        \frac{1}{\sqrt{\cot^4(\theta) + 1}} &= \\
        \sin(\arctan(\tan^2(\theta)))
    \end{align*}

    Hence we can write the above operation as:
    \begin{equation*}
        \begin{pmatrix}
            \cos(\arctan(\tan^2(\theta))) & -i\sin(\arctan(\tan^2(\theta))) \\
            -i\sin(\arctan(\tan^2(\theta))) & \cos(\arctan(\tan^2(\theta))) \\
        \end{pmatrix} =
        e^{-i\arctan(\tan^2(\theta))X}
    \end{equation*}

    In conclusion we get:
    \begin{align*}
        \qket \rightarrow \begin{cases}
            e^{-i\arctan(\tan^2(\theta))X} \qket & \text{if } x = 0, \\
            e^{iX\frac{\pi}{4}} \qket & \text{if } x = 1.
        \end{cases}
    \end{align*}
    Hence, we get what the problem asked us for.

    \subsection*{Part (b)}
    We want to implement an inverse of a $e^{iX\frac{\pi}{4}}$. Hence:
    \begin{align*}
        (e^{iX\frac{\pi}{4}})^\dagger &= \\
        (He^{iZ\frac{\pi}{4}}H)^\dagger &= \\
        (HR_Z(-\frac{\pi}{2})H)^\dagger &= \\
        (HS^\dagger H)^\dagger &= \\
        HSH
    \end{align*}

    \subsection*{Part (c)}
    For $\theta = \frac{\pi}{4}$ we get:
    \begin{align*}
        e^{-i\arctan(\tan^2(\frac{\pi}{4}))X} &= \\
        e^{-i\arctan(1^2)X} &= \\
        e^{-i\frac{\pi}{4}X}
    \end{align*}


\end{homeworkProblem}
\end{document}