\documentclass{article}

\usepackage{fancyhdr}
\usepackage{extramarks}
\usepackage{amsmath}
\usepackage{amsthm}
\usepackage{amsfonts}
\usepackage{tikz}
\usepackage[plain]{algorithm}
\usepackage{algpseudocode}
\usepackage{qcircuit}

\usetikzlibrary{automata,positioning}

%
% Basic Document Settings
%

\topmargin=-0.45in
\evensidemargin=0in
\oddsidemargin=0in
\textwidth=6.5in
\textheight=9.0in
\headsep=0.25in

\linespread{1.1}

\pagestyle{fancy}
\lhead{\hmwkAuthorName}
\chead{}
\rhead{\hmwkClass}
\lfoot{\lastxmark}
\cfoot{\thepage}

\renewcommand\headrulewidth{0.4pt}
\renewcommand\footrulewidth{0.4pt}

\setlength\parindent{0pt}

%
% Create Problem Sections
%

\newcommand{\enterProblemHeader}[1]{
    \nobreak\extramarks{}{Problem \arabic{#1} continued on next page\ldots}\nobreak{}
    \nobreak\extramarks{Problem \arabic{#1} (continued)}{Problem \arabic{#1} continued on next page\ldots}\nobreak{}
}

\newcommand{\exitProblemHeader}[1]{
    \nobreak\extramarks{Problem \arabic{#1} (continued)}{Problem \arabic{#1} continued on next page\ldots}\nobreak{}
    \stepcounter{#1}
    \nobreak\extramarks{Problem \arabic{#1}}{}\nobreak{}
}

\setcounter{secnumdepth}{0}
\newcounter{partCounter}
\newcounter{homeworkProblemCounter}
\setcounter{homeworkProblemCounter}{1}
\nobreak\extramarks{Problem \arabic{homeworkProblemCounter}}{}\nobreak{}

%
% Homework Problem Environment
%
% This environment takes an optional argument. When given, it will adjust the
% problem counter. This is useful for when the problems given for your
% assignment aren't sequential. See the last 3 problems of this template for an
% example.
%
\newenvironment{homeworkProblem}[1][-1]{
    \ifnum#1>0
        \setcounter{homeworkProblemCounter}{#1}
    \fi
    \section{Problem \arabic{homeworkProblemCounter}}
    \setcounter{partCounter}{1}
    \enterProblemHeader{homeworkProblemCounter}
}{
    \exitProblemHeader{homeworkProblemCounter}
}

%
% Homework Details
%   - Title
%   - Due date
%   - Class
%   - Section/Time
%   - Instructor
%   - Author
%

\newcommand{\hmwkTitle}{Homework\ \#1}
\newcommand{\hmwkDueDate}{April 10, 2020}
\newcommand{\hmwkClass}{Intorduction to Quantum Computing}
\newcommand{\hmwkAuthorName}{Jakub Filipek}

%
% Title Page
%

\title{
    % \vspace{in}
    \textmd{\textbf{\hmwkClass:\ \hmwkTitle}}\\
    \normalsize\vspace{0.1in}\small{Due\ on\ \hmwkDueDate\ at 3:10pm}\\
}

\author{\hmwkAuthorName}
\date{}

\renewcommand{\part}[1]{\textbf{\large Part \Alph{partCounter}}\stepcounter{partCounter}\\}

%
% Various Helper Commands
%

% Useful for algorithms
\newcommand{\alg}[1]{\textsc{\bfseries \footnotesize #1}}

% For derivatives
\newcommand{\deriv}[1]{\frac{\mathrm{d}}{\mathrm{d}x} (#1)}

% For partial derivatives
\newcommand{\pderiv}[2]{\frac{\partial}{\partial #1} (#2)}

% Integral dx
\newcommand{\dx}{\mathrm{d}x}

% Alias for the Solution section header
\newcommand{\solution}{\textbf{\large Solution}}

% Probability commands: Expectation, Variance, Covariance, Bias
\newcommand{\E}{\mathrm{E}}
\newcommand{\Var}{\mathrm{Var}}
\newcommand{\Cov}{\mathrm{Cov}}
\newcommand{\Bias}{\mathrm{Bias}}

\newcommand{\norm}[1]{\left\lVert#1\right\rVert}

\begin{document}

\maketitle

% \pagebreak

\begin{homeworkProblem}

    \subsection*{Part (a)}
    \begin{equation*}
        \text{CNOT} = \begin{pmatrix}
            1 & 0 & 0 & 0 \\
            0 & 1 & 0 & 0 \\
            0 & 0 & 0 & 1 \\
            0 & 0 & 1 & 0
        \end{pmatrix}
    \end{equation*}

    \subsection*{Part (b)}
    \[
    \text{SWAP} = \Qcircuit @C=1em @R=.7em {
            & \ctrl{1} & \targ     & \ctrl{1} & \qw \\
            & \targ    & \ctrl{-1} & \targ    & \qw
        }
    \]

    This is because
    \begin{equation*}
        \text{SWAP} = \begin{pmatrix}
            1 & 0 & 0 & 0 \\
            0 & 0 & 1 & 0 \\
            0 & 1 & 0 & 0 \\
            0 & 0 & 0 & 1
        \end{pmatrix}
    \end{equation*}
    And the corresponding circuit is as follows: \\
    \begin{align*}
        \text{CNOT}_{01}\text{CNOT}_{10}\text{CNOT}_{01} &=
        \begin{pmatrix}
            1 & 0 & 0 & 0 \\
            0 & 1 & 0 & 0 \\
            0 & 0 & 0 & 1 \\
            0 & 0 & 1 & 0
        \end{pmatrix}
        \begin{pmatrix}
            1 & 0 & 0 & 0 \\
            0 & 0 & 0 & 1 \\
            0 & 0 & 1 & 0 \\
            0 & 1 & 0 & 0
        \end{pmatrix}
        \begin{pmatrix}
            1 & 0 & 0 & 0 \\
            0 & 1 & 0 & 0 \\
            0 & 0 & 0 & 1 \\
            0 & 0 & 1 & 0
        \end{pmatrix}\\
        &=
        \begin{pmatrix}
            1 & 0 & 0 & 0 \\
            0 & 0 & 1 & 0 \\
            0 & 1 & 0 & 0 \\
            0 & 0 & 0 & 1
        \end{pmatrix}
    \end{align*}

    It is reversible, because CNOTs are reversible. But also because:
    \begin{align*}
        \text{SWAP SWAP} &=\\
        &=
        \begin{pmatrix}
            1 & 0 & 0 & 0 \\
            0 & 0 & 1 & 0 \\
            0 & 1 & 0 & 0 \\
            0 & 0 & 0 & 1
        \end{pmatrix}
        \begin{pmatrix}
            1 & 0 & 0 & 0 \\
            0 & 0 & 1 & 0 \\
            0 & 1 & 0 & 0 \\
            0 & 0 & 0 & 1
        \end{pmatrix} \\
        &=
        \begin{pmatrix}
            1 & 0 & 0 & 0 \\
            0 & 1 & 0 & 0 \\
            0 & 0 & 1 & 0 \\
            0 & 0 & 0 & 1
        \end{pmatrix}
    \end{align*}

    \subsection*{Part (c)}
    To perform such N-bit swap, we not that we can just apply N SWAP gates on $e_{x_i}$ and $e_{y_i}$
    qubits.
    Hence we need to find a implementation of a swap gate with NOT and Toffoli.

    Since the SWAP can be constructed entirely from CNOTs, we this becomes a problem of implementing
    CNOT as combination of Toffoli and NOT gates.

    \[
    \Qcircuit @C=1em @R=.7em {
        & \ctrl{1} & \targ     & \ctrl{1} & \qw &   &
        & \qw      & \ctrl{1}  & \targ     & \ctrl{1}  & \qw & \\
        & \targ    & \ctrl{-1} & \targ    & \qw & = &
        & \qw      & \targ     & \ctrl{-1} & \targ     & \qw & \\
        &          &           &          &     &   &
        & \gate{X} & \ctrl{-1} & \ctrl{-2}  & \ctrl{-1} &     &
    }
    \]

    Note that to implement a N-bit SWAP only one auxiliary qubit is needed,
    and number of Toffoli gate is equal to $3N$.
\end{homeworkProblem}

\vspace{2cm}

\begin{homeworkProblem}
    \begin{table}[h]
        \centering
        \begin{tabular}{ccc|c}
            x & y & c & $x \oplus y$ \\
            \hline
            0 & 0 & 0 & 0 \\
            0 & 1 & 0 & 1 \\
            1 & 0 & 0 & 1 \\
            1 & 1 & 1 & 0 \\
        \end{tabular}
        \caption{Logical table for adding two bits}
    \end{table}

    As we can see $digit = XOR(x, y)$ and $c = AND(x, y)$. Fortunately there exist quantum gates which
    do exactly that:

    \[
    \Qcircuit @C=1.5em @R=2em {
        & \lstick{{|x\rangle}} & \qw & \ctrl{3} & \qw      & \ctrl{2} & \qw  & \rstick{|x\rangle}     & \\
        & \lstick{{|y\rangle}} & \qw & \qw      & \ctrl{2} & \ctrl{1} & \qw  & \rstick{|y\rangle}     & \\
        & \lstick{{|0\rangle}} & \qw & \qw      & \qw      & \targ    & \qw  & \rstick{|c\rangle}     & \\
        & \lstick{{|0\rangle}} & \qw & \targ    & \targ    & \qw      & \qw  & \rstick{|x + y\rangle} &
    }
    \]
\end{homeworkProblem}

\pagebreak
\begin{homeworkProblem}
    \subsection*{Part (a)}
    As mentioned in a lecture all reversible, classical gates, operating on probability distributions
    are just permutations.

    Since permutation does not change the size of the vector, then:
    \begin{equation*}
        |G\psi|_1 = |\psi|_1
    \end{equation*}

    Hence
    \begin{align*}
        \norm{G}_1 &= \\
            &= \max_{|\psi|_1 = 1} |G\psi|_1 \\
            &= \max_{|\psi|_1 = 1} |\psi|_1 \\
            &= 1
    \end{align*}

    \subsection*{Part (b)}

    \begin{align*}
        \norm{\widetilde{G}}_1 &= \\
            &= \max_{|\psi|_1 = 1} |\widetilde{G}\psi|_1 \\
            &= \max_{|\psi|_1 = 1} |((1 - \epsilon)G + \epsilon E)\psi|_1
    \end{align*}
    Since $E$ has only non-negative components and $G$ is a permutation matrix:
    \begin{align*}
        \norm{\widetilde{G}}_1 &= \\
            &= \max_{|\psi|_1 = 1} |(1 - \epsilon)G\psi|_1 + |\epsilon E\psi|_1 \\
            &= \max_{|\psi|_1 = 1} (1 - \epsilon)|G\psi|_1 + \epsilon|E\psi|_1 \\
            &= \max_{|\psi|_1 = 1} (1 - \epsilon) + \epsilon|E\psi|_1 \\
            &= (1 - \epsilon) + \epsilon \max_{|\psi|_1 = 1} |E\psi|_1 \\
            &= (1 - \epsilon) + \epsilon \norm{E}_1
    \end{align*}

    Hence, we can see that:
    \begin{itemize}
        \item If $\norm{E}_1 \ne 1$, then $\norm{\widetilde{G}}_1$ and hence $\widetilde{G}\psi$ is not a
        properly normalized distribution.
        \item If $\norm{\widetilde{G}}_1 = 1$, then $1 = (1 - \epsilon) + \epsilon\norm{\widetilde{E}}_1$
        and hence $\norm{\widetilde{E}}_1 = 1$.
    \end{itemize}

    \subsection*{Part (c)}
    \begin{align*}
        |\widetilde{G}\psi - G\psi|_1 &= \\
            &= |(\widetilde{G} - G)\psi|_1 \\
            &= |(((1 - \epsilon)G + \epsilon E) - G)\psi|_1 \\
            &= |(G - \epsilon G + \epsilon E - G)\psi|_1 \\
            &= |(- \epsilon G + \epsilon E)\psi|_1 \\
            &\le |(\epsilon G + \epsilon E)\psi|_1 \\
            &= |\epsilon G\psi|_1 + |\epsilon E\psi|_1 \\
            &= \epsilon + |\epsilon E\psi|_1 \\
    \end{align*}

    From last part we know that if $\widetilde{G}$ is a valid gate, then $\norm{E}_1 = 1$. Hence
    \begin{align*}
        |\widetilde{G}\psi - G\psi|_1 &\le \\
            &\le \epsilon + |\epsilon E\psi|_1 \\
            &= \epsilon + \epsilon \\
            &= 2\epsilon
    \end{align*}

    \subsection*{Part (d)}
    When we add noise to our \textit{clean} reversible gate of strength $\epsilon$,
    the result will always be within $2\epsilon$ from the truth.

    It shows that there is a linear dependency between noise level of our machine,
    and our certainty in the result.
\end{homeworkProblem}

\vspace{2cm}
\begin{homeworkProblem}
    Hadamard Gate is an example of such operation.

    Let $\psi = |+\rangle$.

    Then $MH\psi = M|0\rangle = |0\rangle$.

    However:
    $HM\psi = H\begin{cases}
        |0\rangle & \text{with } p = 0.5 \\
        |1\rangle & \text{with } p = 0.5 \\
    \end{cases} = 
    \begin{cases}
        |+\rangle & \text{with } p = 0.5 \\
        |-\rangle & \text{with } p = 0.5 \\
    \end{cases}
    $

    The reason for a difference is that quantum computers work on \textit{amplitude}
    of a wavefunction instead its \textit{magnitude}. This allows their amplitudes to cancel completely out,
    which is not possible in classical reversible computing.
\end{homeworkProblem}

\end{document}